%! TEX program = xelatex
\documentclass[9pt,oneside,mpar,cn]{manuscript}

\settheoremfont{\itshape}

\begin{document}

    \title{示例 \handwriting ,这里展示了样式文稿的多数功能}
    \author{\handwriting 许锦文}
    \date{}

    \maketitle

    \vspace{-2\baselineskip}

    \begin{abstract}
        \verb|manuscript|支持英文、中文(\texttt{cn}),并且同时支持\pdfLaTeX 与\XeLaTeX.
    \end{abstract}

    \begin{boxexercise}
   	    \small\tableofcontents
    \end{boxexercise}

    这篇记录主要参考的读物是\cite{manuscript}. 有关\LaTeX 的基本用法,请参阅\cite{latexguide}.

    \vspace{-0.5\baselineskip}
    \section{数学环境}\subtitle{定理、定义、注记、猜想、例子……}

        \begin{theorem}[定理名称] 
            定理的内容.\marginnote{\small environment: \ttfamily theorem}
        \end{theorem}
        \begin{definition}[定义对象]
            具体的定义.\marginnote{\small environment: \ttfamily definition}
        \end{definition}
        \vspace{-0.25\baselineskip}
        \begin{remark}
            一些说明文字.\normalfont\marginnote{\small environment: \ttfamily remark}
        \end{remark}
        \vspace{-0.25\baselineskip}
        \begin{conjecture}[猜想描述]
            具体的内容.\marginnote{\small environment: \ttfamily conjecture}
        \end{conjecture}
        \begin{example}
            一个例子.\marginnote{\small environment: \ttfamily example}
        \end{example}

    \vspace{-0.5\baselineskip}
    \section{页边注}\subtitle{书写想法的绝妙空白}

        如果不想留太大的空白,可以在\verb|\documentclass[...]{manuscript}|中添加\verb|nompar|.
        \begin{itemize*}
            \item 带有数字标记的正式注释\mpar{像这样};\hfill\verb|\mpar{...}|
            \item 带有数字标记的非正式注释\mparhandnum{像这样};\hfill\verb|\mparhandnum{...}|
            \item 普通的非正式注释\mparhand{像这样}.\hfill\verb|\mparhand{...}|
        \end{itemize*}

    \vspace{-0.5\baselineskip}
    \section{模块}\subtitle{适当地分割模块可以让你的思路更加清晰}

        你可以用模块对内容进行分隔,每个模块都有一个序号标记在旁边. \marginnote{\small environment: \ttfamily module}
        \begin{module}
            比如这样. 

            在模块里面还可以有子模块,比如这样:\marginnote{\small environment: \ttfamily submodule}
            \begin{submodule}
                \begin{itemize*}
                    \item 如你所见,子模块是没有编号的,而且颜色很浅
                    \item 使用它可以使层次变得分明
                \end{itemize*}
            \end{submodule}
        \end{module}

    \vspace{-0.5\baselineskip}
    \section{文字样式}\subtitle{对文字进行一些标注有助于突出重点}

    你可以用指定的颜色进行荧光笔标注,例如\highlight{yellow!70}{黄色}\marginnote{\small \ttfamily \textbackslash highlight\{颜色\}\{内容\}};你也可以标注公式:\highlight{yellow!70}{$e^{i\pi}+1=0$}.

    你还可以使用手写字体:{\handwriting 手写},{\enhandwriting\normalsize handwriting}. \marginnote{\small \ttfamily \textbackslash handwriting\\\textbackslash enhandwriting}
    
    \vspace{-0.5\baselineskip}
    \colorem{black}
    \begin{thebibliography}{}
        \bibitem{manuscript} 许锦文. \texttt{manuscript}\textit{样式文稿使用指南}. 版本:2019/5/19.
        \bibitem{latexguide} 许锦文. \textit{一份\emph{\LaTeX}的小指南}. 版本:2019/1/20.
    \end{thebibliography}

\end{document}